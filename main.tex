\documentclass[a4paper, amsfonts, amssymb, amsmath, reprint, showkeys, nofootinbib, twoside]{revtex4-1}
\usepackage[english]{babel}
\usepackage[utf8]{inputenc}
\usepackage[colorinlistoftodos, color=green!40, prependcaption]{todonotes}
\usepackage{amsthm}
\usepackage{mathtools}
\usepackage{physics}
\usepackage{xcolor}
\usepackage{graphicx}
\usepackage{subcaption}
\usepackage[left=23mm,right=13mm,top=35mm,columnsep=15pt]{geometry} 
\usepackage{adjustbox}
\usepackage{placeins}
\usepackage[T1]{fontenc}
\usepackage{lipsum}
\usepackage{csquotes}




\usepackage[pdftex, pdftitle={Article}, pdfauthor={Author}]{hyperref} % For hyperlinks in the PDF
%\setlength{\marginparwidth}{2.5cm}
\usepackage{import}
\usepackage{booktabs}

% tiny scriptsize foottesize small rmalsize large Large LARGE huge Huge




\usepackage{vmargin} %Para modificar los márgenes
\setmargins{2.5cm}{1.5cm}{16.5cm}{23.42cm}{10pt}{1cm}{0pt}{2cm}
\definecolor{ceruleanblue}{rgb}{0.16, 0.32, 0.75}
\usepackage{tocloft}
\renewcommand\cftsecafterpnum{\par\addvspace{6pt}}

\usepackage{booktabs} % toprule midrule bottomrule (lineas en las tablas)
\usepackage{array} 
\usepackage{natbib} % quitar título predeterminado de "The Bibliography"
\usepackage{braket} % Dirac notation
\usepackage{amsmath} % Boxed



\usepackage[pdftex, pdftitle={Article}, pdfauthor={Author}]{hyperref} % For hyperlinks in the PDF
%\setlength{\marginparwidth}{2.5cm}
\usepackage{import}
\usepackage{float} %[h] here,[t] top page,[b] bottom, [H] exactly here (especial del paquete) 
\usepackage{colortbl}
\usepackage{booktabs}
\bibliographystyle{apsrev4-1}
\begin{document}
\renewcommand{\tablename}{\textit{Tabla}}
\renewcommand{\figurename}{\textit{Figura}}

\title{Análisis de árboles de fusiones en simulaciones numéricas}

\author{Víctor Rufo, Yuan -- y Luis Abalo}
    \email[Víctor Rufo, Yuan -- y Luis Abalo: ]{--, -- y labalo@ucm.es}% Your name
    \affiliation{Universidad Complutense de Madrid, Departamento de Astrofísica, Madrid, España}

\date{\today} % Leave empty to omit a date

\begin{abstract}
En el presente artículo se estudia la evolución de los halos de materia oscura utilizando la \textbf{simulación del Milenio} como base de datos. Dicho análisis se divide en dos partes: una primera parte en la que se realiza el estudio de la evolución y las fusiones en halos de rangos de masas diferentes para su posterior comparación. En la segunda parte, con el objetivo de poder estudiar la evolución de las galaxias albergadas en los subhalos, se tendrá en cuenta tanto la materia oscura como la materia bariónica.   
\end{abstract}

\keywords{evolución: galaxias, halos: materia oscura y bariónica, color: excesos, formación: galaxias y estrellas}

\maketitle


\import{sections/}{seccion01.tex} 
\import{sections/}{seccion02.tex}
\import{sections/}{seccion03.tex}
\import{sections/}{seccion04.tex}

\import{sections/}{agradecimientos.tex}


\begin{thebibliography}{6}
\bibitem{1}
\href{https://ui.adsabs.harvard.edu/abs/2005Natur.435..629S/abstract}{V. Springel et al.},
\textbf{The Millennium Simulation}
\textit{Simulation of the formation, evolution and clustering of galaxies and quasars}
(Nature Publishing Group, 435-629, 2005).

\bibitem{2}
\href{https://ui.adsabs.harvard.edu/abs/2007MNRAS.375....2D/abstract}{G. De Lucia and J. Blaizot},
\textit{The hierarchical formation of the brightest cluster galaxies},
Monthly Notices of the Royal Astronomical Society, Volume 375, (2007).

\bibitem{3}
\href{https://ui.adsabs.harvard.edu/abs/2009MNRAS.398.1150B/abstract}{M. Boylan-Kolchin et al.} 
\textbf{The Millennium-II Simulation}
\textit{Resolving cosmic structure formation with the Millennium-II Simulation}
(MNRAS, 398-1150, 2009).

\bibitem{4}
\href{https://ui.adsabs.harvard.edu/abs/2013MNRAS.428.1351G/abstract}{Guo et al.} 
\textbf{The Millennium-WMAP7 Simulation}
\textit{Galaxy formation in WMAP1 and WMAP7 cosmologies}
(MNRAS, 428-1351, 2013).

\bibitem{5}
\href{https://ui.adsabs.harvard.edu/abs/2006astro.ph..8019L/abstract}{G. Lemson and the Virgo Consortium} 
\textbf{Datbase}
\textit{Halo and Galaxy Formation Histories from the Millennium Simulation: Public release of a VO-oriented and SQL-queryable database for studying the evolution of galaxies in the LambdaCDM cosmogony}
(2006)

\bibitem{6}
\href{http://gavo.mpa-garching.mpg.de/Millennium/}{Web: Virgo - Millennium Database}
\end{thebibliography}




% \import{sections/}{apendice01.tex}


\end{document}

