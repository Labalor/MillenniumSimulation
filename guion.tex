% ---------------------------------------------------------------------------------------------------------------------------
% ---------------------------------------------------------------------------------------------------------------------------
\documentclass[8pt, a4paper]{article} %tamaño mínimo de letra 11pto.

% ---------------------------------------------------------------------------------------------------------------------------
% TAMAÑOS DE LETRAS * * * * * * * * * * * * * * * * * * * * * * * * * * * * * * * * * * * * * * * * * * * * * * * * * * * * 
% ---------------------------------------------------------------------------------------------------------------------------
% tiny scriptsize foottesize small rmalsize large Large LARGE huge Huge
% ---------------------------------------------------------------------------------------------------------------------------

% ---------------------------------------------------------------------------------------------------------------------------
% Package including in the initial template * * * * * * * * * * * * * * * * * * * * * * * * * * * * * * * * * * * * * * * *
% ---------------------------------------------------------------------------------------------------------------------------
\usepackage{graphicx} 
\usepackage[spanish]{babel} %Español 
\usepackage[utf8]{inputenc} %Para poder poner tildes
\usepackage{vmargin} %Para modificar los márgenes
\setmargins{2.5cm}{1.5cm}{16.5cm}{23.42cm}{10pt}{1cm}{0pt}{2cm}
% margen izquierdo, superior, anchura del texto, altura del texto, altura de los encabezados, espacio 
% entre el texto y los encabezados, altura del pie de página, espacio entre el texto y el pie de página
% ---------------------------------------------------------------------------------------------------------------------------

% ---------------------------------------------------------------------------------------------------------------------------
% Package  including in the initial template * * * * * * * * * * * * * * * * * * * * * * * * * * * * * * * * * * * * * * * 
% ---------------------------------------------------------------------------------------------------------------------------
\usepackage{float} %[h] here,[t] top page,[b] bottom, [H] exactly here (especial del paquete) 
\usepackage{colortbl}
\definecolor{ceruleanblue}{rgb}{0.16, 0.32, 1}
\usepackage{hyperref}
\hypersetup{
    colorlinks=true,
    linkcolor=blue,
    filecolor=magenta,      
    urlcolor=cyan,
}

\parindent

\begin{document}
\renewcommand{\tablename}{\textit{Tabla}}
\renewcommand{\figurename}{\textit{Figura}}


% ----------------------------------------------------------------------------------------------------------------
\section{\textcolor{ceruleanblue}{Instrucciones del vídeo}}

Análisis de árboles de fusiones en simulaciones numéricas. \\

Consular el artículo \textbf{Springel et al. 2005, Naure 425, 629}. Para la consulta de artículos vamos a utilizar la página de búsqueda de artículos científicos de la NASA: \href{http://adsabs.harvard.edu/abstract_service.html}{NASA website}. Em \textbf{Arxiv} los científicos suelen colgar sus artículos y son de distribución gratuita. También hay que leer el árticulo de \textbf{deLucia y Blaizot (2007, MNRAS 375, 2-4)}, \textbf{Eisenstei y Hut (1998)} y \textbf{Davis et al (1985)}. Describen los modelos semianalíticos que se usan. \\

La simulación que vamos a utilizar es la \textbf{simulación del Millennium}: \href{http://gavo.mpa-garching.mpg.de/Millennium/}{simulación del Millennium} (es bastante antigua pero los resultados son iluminadores respecto a la teoría que demuestran). Es muy importante revisar la \href{http://gavo.mpa-garching.mpg.de/Millennium/Help}{documentación} sobre la base de datos, tanto sobre las características de la simulación (resoluciones, etc) como variables y unidades que proporcionan las \textit{queris}. \\

Respecto a las \textbf{queris}. Las tablas de deLucia contiene los elementos semianalíticos para los bariones. Nos podemos centrar en utilizar las \textbf{demo queris} que para la parte de materia oscura utilizaremos los \textbf{Mainly Halos}: \textbf{H1, H2 y H3}, sobre todo \textbf{H1 y H2} y para la parte bariónica los \textbf{Mainly Galaxies}: sujetas a los índices obtenidos en los halos de materia oscura, ya que lo que se nos pide es relacionar estos halos de materia oscura obtenidos con las queris H con las obtenidas con las queris G. \\

El \textbf{formato de queris}. Para las queris H nos vamos a la librería MPAHalo (Librería de Halos de Materia Oscura). Podemos elegir el momento del snapshot, entre qué número de partículas (np) tenga cada halo que saldrá en la queri. También podemos hacer la queri en masa (ver documentación, será uno de los ejercicios que se proponga). Para reducir el tiempo de espera, se puede poner en qué rango de la caja cosmológica busca. \\

Tenemos información sobre las \textbf{generic queries}. Por ejemplo, para las de \textbf{H1} se pueden coger para un redshit concreto. Para el redshit 0 está con el snapnum$=64$, en el que tendremos que centrarnos para escoger halos según su masa (se hace con H1). Recordad: me da todos los halos a un redshit determinado. \\

La queri \textbf{H2} me da, para un halo concreto (DES.haloID) me da todo el árbol de fusiones. Primero, evolución del halo principal para todos los snapshots hasta llegar a su primera fusión, a partir de donde tendremos los halos secundarios. Vamos a obtener una evolución de cada uno de los halos que se van a ir fusionando hasta llegar a tener el halo final a redshit cero. \\

La quieri \textbf{H3} sirve para obtener los progenitores a un redshift determinado. Pero básicamente nos centraremos en H1 y H2 en este trabajo. \\

Es \textbf{importante} poner, en h2, un número infinito de líneas que queremos que nos regrese la queri porque queremos todo el árbol de fusiones. El resultado será una tabla con las distintas variables para cada uno de los halos que estamos viendo. Ver la documentación para ver qué significa cada columna. Por ejemplo: haloID (ID del halo principal), subHaloID (ID del halo secundario), lastProgenitor (último progenitor), descendant (descenciente principal), snapNum, redshift del snapshot, mMean200 (masa media 200), np (número de partículas del halo), mCri200 (masa crítica 200), mTopHat (masa del Top Hat)... \\

Para la selección de halos de un rango de masas tenemos que usar la masa típica de cosmología, que es la \textbf{masa típica 200}, que es la masa que queda dentro de una esfera que tiene una densidad igual a 200 veces la densidad crítica del Universo. \\

\newpage
\subsection{Objetivos}

\textbf{Objetivo 1: obtener los gráficos de la Figura 1}\\ 

\textbf{Obtener un árbol de fusiones para halos de materia oscura de diferentes masas totales. Hacerlo primero para halos individuales y luego obtener la historia de fusiones promedio (ver ala van den Bosch 2002)}. Nos propone hacerlo para 50 halos distribuidos de la siguiente forma: 

\begin{itemize}
\item 10 halos en el rango de masas mayores a $10^{13}M_{\odot}$   
\item 10 halos para el rango de masas $10^{13}M_{\odot}-10^{12}M_{\odot}$
\item 10 halos para el rango de masas $10^{12}M_{\odot}-10^{11}M_{\odot}$
\item 10 halos para el rango de masas $10^{11}M_{\odot}-10^{10}M_{\odot}$
\item 10 halos para el rango de masas menores a $10^{10}M_{\odot}$
\end{itemize}

Nos recomienda el rango segundo y el tercero. En los siguientes objetivos nos recomienda rangos entre 13 y 12, y porque tendremos que discutir qué es lo que vemos en el rango entre 12 y 11. \\
Tendremos que obtener algo parecido a las gráficas de la Figura 1. 

\begin{figure}[H]
\centering
\includegraphics[width=1\textwidth]{{figura1}.png}
\caption{Figura 1}
\end{figure}

A la izquierda, un modelo de simulación y a la derecha, el modelo teórico que podremos encontrar en los árticulos. Para obtener la figura de la izquierda hay que pesar cada uno de los halos por su masa final (redshift cero) porque sino no vamos a obtener una convergencia en el punto de arriba a la izquierda. Es decir, que la $M_0$ tiene que ser la masa final para cada uno de los halos. Y hay que hacerlo para cad auno de los cinco rangos. \\

Posteriormente hay que hacerlo para las queris G1 y para la componente de masa bariónica. Ver como crece la materia bariónica con el tiempo y compararlo con los modelos semianalíticos. \\  

\textbf{Objetivo 2:}\\

\textbf{Analizar los resultados comparándolos con la masa estelar predicha por los modelos semianalíticos (deLucia, etc). Hacerlo tanto para los precursores como para la galaxia resultante (redshift cero).} \\

\textbf{Objetivo 3:}\\

\textbf{Determinar la evolución del color (B-V) promedio de las galaxias principales en el árbol, en función de la masa estelar final, de la masa total y de la dispersión de velocidades.} \\
Hay que seleccionar tres halos. Masa superior a $10^{12}M_{\odot}$ porque será mucho más interesante y el análisis más profundo. Habrá que obtener el árbol para la materia oscura y para la componente bariónica, y analizar, para ambos, sus propiedades. Para el caso específico de la componente bariónica, el índice de color, la dispersión de velocidades y la masa total. Hay que obtener algo parecido a la Figura 2. \\

\begin{figure}[H]
\centering
\includegraphics[width=1\textwidth]{{figura2}.png}
\caption{Figura 2}
\end{figure}

Nos aconseja Python (librerías de pandas y matplotlib). Hay que hacer un artículo científico en LaTeX explicando la descripción y las conclusiones: introducción con el contexto teórico, descripción de la simulación y hablar de los tres resultados (1: halos de distintos rangos de masas, 2: discusión y comparación con los modelos teóricos, 3: árboles de fusión y discusión de dichos árboles). Además, podemos ampliar el análisis tanto de la primera parte como la del árbol de fusiones. La premisa básica es que tendremos un ocho si hacemos lo que pide el guión. Para llegar al 10 hay que revisar bibliografía extra, hacer análisis extra e interpretrarlo. Hay que poner todas las referencias que se consulten. \\

Señalar que nos vamos a encontrar limitaciones, sobre todo a la hora de hacer los gráficos. No quedarán los gráficos como en los árticulos, así que tendremos que ir preguntándole. \\

\section{Detalles durante las clases prácticas}

\begin{itemize}
\item Estudiar la dpendencias del redshift es para nota 
\end{itemize}     











% ---------------------------------------------------------------------------------------------------------------------------
\end{document}

