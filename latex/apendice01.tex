\section*{Apéndice} 
\label{sec:apendice} % no poner acentos

\begin{figure*}
    \centering
    \begin{minipage}{0.45\textwidth}
        \centering
        \includegraphics[width=0.9\textwidth]{{1000}.png} 
	 \subcaption{Evolución de los 10 halos del \textit{Rango 1}}
    \end{minipage}
    \begin{minipage}{0.45\textwidth}
        \centering
        \includegraphics[width=0.9\textwidth]{{100}.png} 
        \ \subcaption{Evolución de los 10 halos del \textit{Rango 2}}
    \end{minipage}
    \begin{minipage}{0.45\textwidth}
        \centering
        \includegraphics[width=0.9\textwidth]{{10}.png} 
         \subcaption{Evolución de los 10 halos del \textit{Rango 3}}
    \end{minipage}
    \begin{minipage}{0.45\textwidth}
        \centering
        \includegraphics[width=0.9\textwidth]{{1}.png}
         \subcaption{Evolución de los 10 halos del \textit{Rango 4}}
    \end{minipage}
    \begin{minipage}{0.45\textwidth}
        \centering
        \includegraphics[width=0.9\textwidth]{{0}.png} 
         \subcaption{Evolución de los 10 halos del \textit{Rango 5}}
    \end{minipage}
    \caption{\textbf{Evolución de los halos} referentes a los rangos de la \textit{Tabla} ~\ref{tab:tabla2} para el caso de materia oscura. Se explica en la \textit{Sección}~\ref{subsec:2_B}}
    \label{fig:fig3}
\end{figure*}

\begin{figure*}
\centering
\includegraphics[width=1\textwidth]{{AveragesSubplot}.png} 
\caption{\textbf{Desviación típica} para el mismo tiempo en la evolución de los halos de la \textit{Figura}~\ref{fig:fig1} para la meteria oscura. Se explica en la \textit{Sección}~\ref{subsec:2_B}}
\label{fig:fig4}
\end{figure*}


\begin{figure*}
    \centering
    \begin{minipage}{0.45\textwidth}
        \centering
        \includegraphics[width=0.9\textwidth]{{Baryon_1000}.png} 
	 \subcaption{Evolución de los 10 halos del \textit{Rango 1}}
\label{fig:figura1} 
    \end{minipage}
    \begin{minipage}{0.45\textwidth}
        \centering
        \includegraphics[width=0.9\textwidth]{{Baryon_100}.png} 
        \ \subcaption{Evolución de los 10 halos del \textit{Rango 2}}
    \end{minipage}
    \begin{minipage}{0.45\textwidth}
        \centering
        \includegraphics[width=0.9\textwidth]{{Baryon_10}.png} 
         \subcaption{Evolución de los 10 halos del \textit{Rango 3}}
    \end{minipage}
    \begin{minipage}{0.45\textwidth}
        \centering
        \includegraphics[width=0.9\textwidth]{{Baryon_1}.png}
         \subcaption{Evolución de los 10 halos del \textit{Rango 4}}
    \end{minipage}
    \begin{minipage}{0.45\textwidth}
        \centering
        \includegraphics[width=0.9\textwidth]{{Baryon_0}.png} 
         \subcaption{Evolución de los 10 halos del \textit{Rango 5}}
    \end{minipage}
    \caption{\textbf{Evolución de los halos} referentes a los rangos de la \textit{Tabla} ~\ref{tab:tabla2} para el caso de materia bariónica. Se explica en la \textit{Sección}~\ref{subsec:2_C}}
\label{fig:fig5}
\end{figure*}

\begin{figure*}
\centering
\includegraphics[width=1\textwidth]{{AveragesSubplot_baryons}.png} 
\caption{\textbf{Desviación típica} para el mismo tiempo en la evolución de los halos de la \textit{Figura}~\ref{fig:fig2} para la meteria bariónica. Se explica en la \textit{Sección}~\ref{subsec:2_C}}
\label{fig:fig6}
\end{figure*}

\begin{figure*}
    \centering
    \begin{minipage}{0.45\textwidth}
        \centering
        \includegraphics[width=0.9\textwidth]{{FussionTree2_G100_1}.png} 
	 \subcaption{Evolución de los 10 halos del \textit{Rango 1}}
    \end{minipage}
    \begin{minipage}{0.45\textwidth}
        \centering
        \includegraphics[width=0.9\textwidth]{{FussionTree2_G100_2}.png} 
        \ \subcaption{Evolución de los 10 halos del \textit{Rango 2}}
    \end{minipage}
    \begin{minipage}{0.45\textwidth}
        \centering
        \includegraphics[width=0.9\textwidth]{{FussionTree2_G100_8}.png} 
         \subcaption{Evolución de los 10 halos del \textit{Rango 3}}
    \end{minipage}
    \begin{minipage}{0.45\textwidth}
        \centering
        \includegraphics[width=0.9\textwidth]{{FussionTree2_G100_9}.png}
         \subcaption{Evolución de los 10 halos del \textit{Rango 4}}
    \end{minipage}
    \caption{Representación del \textbf{árbol de fusiones} para los cuatro subhalos escogidos. Pertenecen al Rango 4 de la \textit{Tabla} ~\ref{tab:tabla2}. Se explica en la \textit{Sección}~\ref{subsec:3_B}}
    \label{fig:fig7}
\end{figure*}

\begin{figure*}
    \centering
    \begin{minipage}{0.45\textwidth}
        \centering
        \includegraphics[width=0.9\textwidth]{{FussionTree_G100_1}.png} 
	 \subcaption{Evolución de los 10 halos del \textit{Rango 1}}
    \end{minipage}
    \begin{minipage}{0.45\textwidth}
        \centering
        \includegraphics[width=0.9\textwidth]{{FussionTree_G100_2}.png} 
        \ \subcaption{Evolución de los 10 halos del \textit{Rango 2}}
    \end{minipage}
    \begin{minipage}{0.45\textwidth}
        \centering
        \includegraphics[width=0.9\textwidth]{{FussionTree_G100_8}.png} 
         \subcaption{Evolución de los 10 halos del \textit{Rango 3}}
    \end{minipage}
    \begin{minipage}{0.45\textwidth}
        \centering
        \includegraphics[width=0.9\textwidth]{{FussionTree_G100_9}.png}
         \subcaption{Evolución de los 10 halos del \textit{Rango 4}}
    \end{minipage}
    \caption{Representación del \textbf{árbol de fusiones} para los cuatro subhalos escogidos (la masa queda reflejada en el tamaño de los puntos). Pertenecen al Rango 4 de la \textit{Tabla} ~\ref{tab:tabla2}. Se explica en la \textit{Sección}~\ref{subsec:3_B}}
    \label{fig:fig8}
\end{figure*}

\begin{figure*}
    \centering
    \begin{minipage}{0.45\textwidth}
        \centering
        \includegraphics[width=0.9\textwidth]{{FussionTree3_G100_1}.png} 
	 \subcaption{Evolución de los 10 halos del \textit{Rango 1}}
    \end{minipage}
    \begin{minipage}{0.45\textwidth}
        \centering
        \includegraphics[width=0.9\textwidth]{{FussionTree3_G100_2}.png} 
        \ \subcaption{Evolución de los 10 halos del \textit{Rango 2}}
    \end{minipage}
    \begin{minipage}{0.45\textwidth}
        \centering
        \includegraphics[width=0.9\textwidth]{{FussionTree3_G100_8}.png} 
         \subcaption{Evolución de los 10 halos del \textit{Rango 3}}
    \end{minipage}
    \begin{minipage}{0.45\textwidth}
        \centering
        \includegraphics[width=0.9\textwidth]{{FussionTree3_G100_9}.png}
         \subcaption{Evolución de los 10 halos del \textit{Rango 4}}
    \end{minipage}
    \caption{Representación 3D del \textbf{árbol de fusiones} para los cuatro subhalos escogidos. Pertenecen al Rango 4 de la \textit{Tabla} ~\ref{tab:tabla2}. Se explica en la \textit{Sección}~\ref{subsec:3_B}}
    \label{fig:fig9}
\end{figure*}


