\section{Evolución de halos para distintas masas} 
\label{sec:2} % no poner acentos

\subsection{Procedimiento}
\label{subsec:2_A}

En esta primera parte, se lleva a cabo el estudio de la evolución de los árboles de fusión de halos de materia oscura para cinco rangos de masa (ver \textit{Tabla} ~\ref{tab:tabla2}) a lo largo de su historia. Posteriormente, se analiza el promedio de la evolución de los halos para cada rango de masas. Así, para cada rango se analiza la evolución de 10 halos, representando su historia de fusiones y su crecimiento en masa respecto al \textit{redshift}.   \\

\begin{table}[H]
\begin{center}
\begin{tabular}{ccc}
\toprule
\cellcolor[gray]{0.9}Rango 1 & & $M_{crit}>10^{13}M_\odot$ \\
\midrule
\cellcolor[gray]{0.9}Rango 2 & & $10^{12}M_\odot<M_{crit}<10^{13}M_\odot$ \\
\midrule
\cellcolor[gray]{0.9}Rango 3 & & $10^{11}M_\odot<M_{crit}<10^{12}M_\odot$ \\
\midrule
\cellcolor[gray]{0.9}Rango 4 & & $10^{10}M_\odot<M_{crit}<10^{11}M_\odot$ \\
\midrule
\cellcolor[gray]{0.9}Rango 5 & & $10^{10}M_\odot>M_{crit}$ \\
\bottomrule
\end{tabular}
\end{center}
\caption{Rangos de masas a tiempo actual con \textit{redshift} nulo}
\label{tab:tabla2}
\end{table}

Para obtener la evolución de los halos, se trabaja con los modos H1 y H2 de la base de datos del \textit{Millennium} \cite{6}. El modo H1 se emplea para situar los halos con una masa crítica determinada en el rango correspondiente. Posteriormente, con el modo H2, se obtiene la historia de formación completa. \\

Utilizando el modo H1, se establece un \textit{redshift} 0 mediante un \textit{Snapnum = 63}. La determinación de los intervalos para la masa se lleva a cabo variando en número de partículas \textit{np} y el tamaño de la caja \textit{(x,y,z)}. Se ha de tener en cuenta que la masa viene nada en unidades de $10^{10}M_\odot$. \\

Seguidamente, se introduce el \textit{HaloID} utilizando el modo H2. Los datos obtenidos contienen la evolución del halo en cuestión, junto a sus fusiones correspondientes. Estos datos se guardan en un fichero $.csv$ y se almacenan en una carpeta para su posterior tratamiento. El estudio se centra en el halo principal, al cual otros halos menores se le han ido fusionando a lo largo de la historia. \\

Los datos almacenados han de ser procesados, pues se tienen que eliminar los errores como consecuencia de las limitaciones de la simulación. \\
% Añadir aquí la explicación de la eliminación de los picos. 
Finalmente, se tienen los archivos preparados para su representación gráfica. \\

\subsection{Resultados}
\label{subsec:2_B}

Una vez los ficheros $.csv$ están corregidos, se representa gráficamente cada rango de masa de la \textit{Tabla} ~\ref{tab:tabla2} de la siguiente manera. Se dispone la masa crítica de cada halo normalizada con la masa a \textit{redshift} nulo en frente del inverso del factor de escala, ambos ejes en escala logarítmica decimal. \\  

\subsubsection{Evolución de los halos del Rango 1 (ver \textit{Tabla} ~\ref{tab:tabla2})}

Se representan los halos correspondientes al Rango 1 de masas: $M_{crit}>10^{13}M_\odot$, \textit{Tabla} ~\ref{tab:tabla2}. La masa crítica, $m_{crit200}$, de cada \textit{HaloID} correspondiente, se recogen en la \textit{Tabla} ~\ref{tab:tabla3} \\ 

\begin{figure}[H]
\centering
\includegraphics[width=0.55\textwidth]{{prueba_100}.png}
\caption{\textbf{Prueba} Evolución de los 10 halos del Rango 1 de la \textit{Tabla} ~\ref{tab:tabla2}}
\label{fig:figura1} 
\end{figure}

% \subsubsection{Evolución de los halos del Rango 2 (ver \textit{Tabla} ~\ref{tab:tabla2})}

% \subsubsection{Evolución de los halos del Rango 3 (ver \textit{Tabla} ~\ref{tab:tabla2})}

% \subsubsection{Evolución de los halos del Rango 4 (ver \textit{Tabla} ~\ref{tab:tabla2})}

% \subsubsection{Evolución de los halos del Rango 5 (ver \textit{Tabla} ~\ref{tab:tabla2})}


























