\section{Evolución de halos para distintas masas} 
\label{sec:2} % no poner acentos

\subsection{Procedimiento}
\label{subsec:2_A}

\begin{table}[H]
\begin{center}
\begin{tabular}{ccc}
\toprule
\cellcolor[gray]{0.9}Rango 1 & & $M_{crit200}>10^{13}M_\odot$ \\
\midrule
\cellcolor[gray]{0.9}Rango 2 & & $10^{12}M_\odot<M_{crit200}<10^{13}M_\odot$ \\
\midrule
\cellcolor[gray]{0.9}Rango 3 & & $10^{11}M_\odot<M_{crit200}<10^{12}M_\odot$ \\
\midrule
\cellcolor[gray]{0.9}Rango 4 & & $10^{10}M_\odot<M_{crit200}<10^{11}M_\odot$ \\
\midrule
\cellcolor[gray]{0.9}Rango 5 & & $10^{10}M_\odot>M_{crit200}$ \\
\bottomrule
\end{tabular}
\end{center}
\caption{\textbf{Rangos de masas} de los halos a \textit{redshift} cero.}
\label{tab:tabla2}
\end{table}

En esta primera parte, se lleva a cabo el estudio de la evolución en masa de los árboles de fusión de halos de materia oscura para cinco rangos de masa (ver \textit{Tabla}~\ref{tab:tabla2}) a lo largo de su historia. Posteriormente, se analiza el promedio de la evolución de los halos para cada rango de masas. Así, para cada rango se analiza la evolución de 10 halos, representando su historia de fusiones y su crecimiento en masa respecto al \textit{redshift}.   \\

Para obtener la evolución de los halos, se trabaja con los modos H1 y H2 de la base de datos del \textit{Millennium} \cite{6}. El modo H1 se emplea para situar los halos con una masa crítica determinada en el rango correspondiente. Posteriormente, con el modo H2, se obtiene la historia de formación completa. \\

Utilizando el modo H1, se establece un \textit{redshift} 0 mediante un \textit{Snapnum = 63}. La determinación de los intervalos para la masa se lleva a cabo variando el intervalo de la masa crítica 200 $(m_{Crit200})$ y el tamaño de la caja \textit{(x,y,z)}. Se ha de tener en cuenta que la masa viene dada en unidades de $10^{10}M_\odot$. El código desarrollado se almacena en un repositorio online \cite{9}. \\

Seguidamente, se introduce el \textit{HaloID} utilizando el modo H2. Los datos obtenidos contienen la evolución del halo en cuestión, junto a sus fusiones correspondientes. Estos datos se guardan en un fichero $.csv$ y se almacenan en una carpeta para su posterior tratamiento. El estudio se centra en la rama principal del halo, a la cual otros halos menores se le han ido fusionando a lo largo de la historia. \\

Se representa gráficamente cada rango de masa de la \textit{Tabla}~\ref{tab:tabla2} de la siguiente manera. Se dispone la masa crítica de cada halo normalizada con la masa a \textit{redshift} cero en frente del inverso del factor de escala, ambos ejes en escala logarítmica decimal. 

\subsection{Resultados: Materia oscura}
\label{subsec:2_B}

En primer lugar, cabe señalar que el crecimiento de los halos se realizan mediante fusiones y, en la simulación, hay una discretización temporal. Por tanto, el crecimiento ocurre en un instante de tiempo $\Delta_t$, es decir, en un \textit{snapNum} determinado, y no tiene un crecimiento continuo. Sin embargo, debido a que la representación escalonada de los subhalos dificulta la visualización, en la \textit{Figura}~\ref{fig:fig3} no se añade. \\

El objetivo de evidenciar la dependencia con la masa en la historia de los halos de materia oscura, se ha presentado la \textit{Figura}~\ref{fig:fig1} a modo de comparación de las medias de los halos pertenecientes a cada intervalo de masas

De manera previa a estudiar la evolución de los subhalos para cada rango de masa en la \textit{Figura}~\ref{fig:fig3}, y con el objetivo de evidenciar la dependencia con la masa en la historia de los halos de materia oscura, se presenta en la \textit{Figura}~\ref{fig:fig1} una comparación general entre los cinco rangos de masas. En esta figura, se calcula la media de las masas de los subhalos de cada rango y para cada redshift respecto a las masas finales a redshift nulo. Las escalas de los ejes son logarítmicas, luego cada intervalo representa la variación en órdenes de magnitud. \\

\begin{figure}[H]
\centering
\includegraphics[width=0.5\textwidth]{{Averages}.png} 
\caption{Promedio de la \textbf{evolución en masa de los halos} para cada rango de masa  (\textit{Tabla}~\ref{tab:tabla2}) para el caso de materia oscura. Código de colores: morado-Rango 1, rojo-Rango 2, verde-Rango 3, amarillo-Rango 4 y azul-Rango 5}
\label{fig:fig1}
\end{figure}

Se observa claramente que los halos más masivos tienen una historia más antigua al llegar a redshifts más elevados. Hay un crecimiento inicial considerable y más homogéneos en los tres rangos más masivos y un aplanamiento de su curva en redshift más cercanos a cero. Mientras tanto, las curvas que representan los rangos menos masivos no llegan a altos redshifts. Las limitaciones del modelo se ponen en evidencia en el rango menos masivo, al presentar una curva nada suave y en la que la masa principalmente decrece en los halos, en lugar de crecer como ocurre en el resto de rangos. Se hablará de las limitaciones de la simulación del \textbf{Milenio}\cite{6} en la \textit{Sección}~\ref{sec:4}. \\

Aunque la tendencia general es un crecimiento en la masa, que se corresponde con el crecimiento jerárquico por las fusiones de halos, se observan también caídas en la masa. Estas caídas son menos frecuentes en los rangos de mayor masa, pero son dominantes en el rango de masas menores. Las que ocurren en halos de mayor masa pueden deberse a la salida de partículas del radio 200 (radio en que la sobredensidad es 200 veces la crítica de la simulación) tras fusiones violentas, lo que hace que esa parte de la masa deje de tenerse en cuenta hasta que la gravedad la vuelve a introducir en el radio 200, o por interacciones con halos mayores que les quiten masa. En el caso de los halos del rango más bajo de masas, aunque parte de estos decrecimientos se deben a que las interacciones con halos mayores perdiendo masa son más frecuentes, el hecho de que sea una tendencia general se debe a las limitaciones de la simulación, ya que se están seleccionando halos muy cercanos al límite que el algoritmo Subfind (Springel et al., 2001) \cite{8} establece para considerar un grupo de partículas como un subhalo. Es decir, no se pueden obtener subhalos de estas masas que provengan de otros menores, ya que estos progenitores no se considerarían subhalos; solo se tienen subhalos en cuya historia reciente fueron mayores y perdieron masa. Por eso mismo, no se puede conocer su historia a alto redshift. \\

\begin{comment}
La \textit{Figura}~\ref{fig:fig1} representa la evolución de los halos para cada rango de la \textit{Tabla}~\ref{tab:tabla2}. En general, se observan intervalos de crecimiento negativo, lo cuál no tiene una explicación física congruente. Estos decrecimientos son una consecuencia de las limitaciones del modelo y se explican entendiendo que, para un \textit{SnapNum} determinado, halos más pequeños salen fuera del radio 200 del halo principal (radio en que la sobredensidad es 200 veces la crítica de la simulación), dejando de ser considerados, sustrayendo así la masa correspondiente. Posteriormente, este decrecimiento se recupera gracias al colapso gravitacional. Los halos que habían salido vuelven a incluirse en el espacio que abarca el radio del virial del halo principal. \\
\end{comment}

Si uno se detiene en cada gráfica dentro de la \textit{Figura}~\ref{fig:fig3} (Apéndice), se pueden hacer los siguientes comentarios:
\begin{itemize}
\item En \textit{(a)}, se aprecia un crecimiento inicial a alto redshift con un número considerable de fusiones entre los distintos subhalos, visualizando un aplanamiento de la curva hacia redshifts más pequeños. Esto se debe a que, cuanto mayor es el halo, menor diferencia de masa relativa producen las fusiones. 
\item En \textit{(b)} y en \textit{(c)}, se puede verificar que el crecimiento inicial es menor al fijarse en el eje de abscisas, cuyos intervalos se han reducido, por lo que la curva tiene un aplanamiento más temprano. La principal diferencia entre \textit{(b)} y \textit{(c)} es que en el tercer rango de masas, no se llega a un redshift tan elevado. 
\item En \textit{(d)}, fijándose de nuevo en el eje de abscisas, el crecimiento es mucho más tardío y menos acentuado. 
\item Finalmente, en (e), se observa una evolución muy irregular que no se suaviza para redshifts cercanos a 0, algo, que por el contrario, es común al resto de rangos de masa. Esto es una consecuencia de la limitación de la simulacion. Esta variación abrupta de masa se debe a que, como son partículas muy masivas, cuando se define un halo de baja masa, este está dado por pocas partículas. De esta forma, la pérdida de pocas partículas, ya sea por interacciones o porque estas salgan del radio 200, supone una gran pérdida de masa relativa. Esto reduce siginificativamente la posibilidad de una estabilidad en el comportamiento de la curva para ningún valor del redshift.
\end{itemize}
\\

La figura \textit{Figura}~\ref{fig:fig4} (Apéndice) presenta la desviación típica relacionada con cada una de las medias realizadas para la \textit{Figura}~\ref{fig:fig1}. Esto permite conocer la validez de dichas medias y conocer si la evolución de los halos dentro de un rango de masas son muy diferentes entre sí (grandes desviaciones) o si, por el contrario, su comportamiento es similar (bajas desviaciones). En todos los casos, las desviaciones tienden a cero en el presente, lo que es lógico, ya que cada halo tiene su propia masa en redshift nulo y en poco tiempo no puede cambiar mucho. En algunos casos, también se hacen cero a redshifts muy altos, esto es debido a que hay menos halos sobre los que promediar o incluso uno solo, ya que a mayores redshifts los halos son tan pequeños que no son considerados por el algoritmo Subfind (Springel et al., 2001) \cite{8} y no se tiene información de ellos. Para redshifts intermedios es cuando esta información es interesante; la desviación típica es menor en los rangos de mayor masa, ya que todos los halos tienen un crecimiento suave similar, mientras que los rangos de masas menores muestran una desviación mayor, ya que la variación en las evoluciones es muy grande por los efectos que ya se han comentado. \\

\begin{comment}
La figura \textit{Figura}~\ref{fig:fig4} presenta el promedio de las masas de los halos para los cinco intervalos de masa de la \textit{Tabla}~\ref{tab:tabla2}. Esta gráfica, refleja un cambio portencual respecto al estado final a redshift cero. Por tanto, para un redhift dado se ve que un cierto halo ha ganado o perdido un cierto porcentaje de su masa final. Cabe esperar que para redshift cercanos al 0, su cambio porcentual sea muy pequeño, es decir, converja al estado final de manera suave. El halo ha llegado a una etapa estable y definida. Si ahora se implementa el concepto de desviación típica con muchos halos con los que promediar, ésta sera mínima para esta etapa final y máxima para el estado inicial. Esto se debe a que cada halo parte con comportamientos másicos muy distintos. Si se observa alguna desviación nula comprendida entre dos momentos de desviacin alta a redshift altos, esto se debe a que únicamente un halo con el que se ha promediado llega a ese redhisft. No quiere decir, por tanto, que todos los halos converjan en ese punto, si no que solo hay uno. \\

Es decir, se interpreta la desviación como la diferencia máxima entre un cambio porcentual de un halo respecto a otro para un redshift determinado. Conforme mayor sea la desviación, más distintos son los comportamientos. Los halos ganan o pierden masa de una manera más arbitraria. Si es materia negra, esto se debe a un intercambio por interacción gravitatoria. Ahota bien, si la desviación se reduce, los halos cambian de manera similar. Son más estables, tendrán menos interacciones. \\
\end{comment}

De esta manera, se interpretan mejor las gráficas, resaltando la mala resolución que se produce en los halos de baja masa, cuya desviación es grande de manera continuada. 

\subsection{Resultados: Materia bariónica}
\label{subsec:2_C}

De la misma manera que en la \textit{Sección} ~\ref{subsec:2_B}, el crecimiento es instantáneo pero, para una mejor visualización, la \textit{Figura} ~\ref{fig:fig5} no se representa de manera escalonada. \\

Análogamente, antes de analizar la \textit{Figura}~\ref{fig:fig5} en la que se recogen la evolución de cada conjunto de subhalos para cada rango de masas de la \textit{Tabla}~\ref{tab:tabla2}, se presenta en la \textit{Figura}~\ref{fig:fig2} un promedio de las masas para cada rango y para cada redshift. Igualmente, las esalas son logarítmicas, con lo que la variación entre cada intervalo corresponde a un cambio en el orden de magnitud. \\

El crecimiento de los halos más masivos se prolonga hacia redshift más grandes, por lo tanto su historia de fusiones es más antigua. El aplanamiento de la curva de crecimiento según se avanza hacia redshift más pequeños también es visible en la gráfica. Y, de la misma manera que en la \textit{Figura}~\ref{fig:fig1}, las limitaciones del modelo se hacen visible en la curva azul, correspondiente al promedio del grupo de subhalos del rango más pequeño de masa. Sin embargo, la principal diferencia con el caso de la materia bariónica de la \textit{Sección}~\ref{subsec:2_B}, es que el crecimiento no es igual de homogéneo. La explicación se contará en detalle con la explicación relativa a la \textit{Figura}~\ref{fig:fig5}. \\

\begin{figure}[H]
\centering
\includegraphics[width=0.5\textwidth]{{Averages_baryons}.png} 
\caption{Promedio de la \textbf{evolución de los halos} para cada rango de masa  (\textit{Tabla} ~\ref{tab:tabla2}) para el caso de materia bariónica. Código de colores: morado-Rango 1, rojo-Rango 2, verde-Rango 3, amarillo-Rango 4 y azul-Rango 5}
\label{fig:fig2}
\end{figure}















