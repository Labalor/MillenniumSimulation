\section{Evolución de halos para distintas masas} 
\label{sec:2} % no poner acentos

\subsection{Procedimiento}
\label{subsec:2_A}

\begin{table}[H]
\begin{center}
\begin{tabular}{ccc}
\toprule
\cellcolor[gray]{0.9}Rango 1 & & $M_{crit}>10^{13}M_\odot$ \\
\midrule
\cellcolor[gray]{0.9}Rango 2 & & $10^{12}M_\odot<M_{crit}<10^{13}M_\odot$ \\
\midrule
\cellcolor[gray]{0.9}Rango 3 & & $10^{11}M_\odot<M_{crit}<10^{12}M_\odot$ \\
\midrule
\cellcolor[gray]{0.9}Rango 4 & & $10^{10}M_\odot<M_{crit}<10^{11}M_\odot$ \\
\midrule
\cellcolor[gray]{0.9}Rango 5 & & $10^{10}M_\odot>M_{crit}$ \\
\bottomrule
\end{tabular}
\end{center}
\caption{\textbf{Rangos de masas} de los halos a \textit{redshift} nulo.}
\label{tab:tabla2}
\end{table}

En esta primera parte, se lleva a cabo el estudio de la evolución en masa de los árboles de fusión de halos de materia oscura para cinco rangos de masa (ver \textit{Tabla} ~\ref{tab:tabla2}) a lo largo de su historia. Posteriormente, se analiza el promedio de la evolución de los halos para cada rango de masas. Así, para cada rango se analiza la evolución de 10 halos, representando su historia de fusiones y su crecimiento en masa respecto al \textit{redshift}.   \\

Para obtener la evolución de los halos, se trabaja con los modos H1 y H2 de la base de datos del \textit{Millennium} \cite{6}. El modo H1 se emplea para situar los halos con una masa crítica determinada en el rango correspondiente. Posteriormente, con el modo H2, se obtiene la historia de formación completa. \\

Utilizando el modo H1, se establece un \textit{redshift} 0 mediante un \textit{Snapnum = 63}. La determinación de los intervalos para la masa se lleva a cabo variando el intervalo de la masa crítica 200 \textit{m_Crit200} y el tamaño de la caja \textit{(x,y,z)}. Se ha de tener en cuenta que la masa viene nada en unidades de $10^{10}M_\odot$. \\

Seguidamente, se introduce el \textit{HaloID} utilizando el modo H2. Los datos obtenidos contienen la evolución del halo en cuestión, junto a sus fusiones correspondientes. Estos datos se guardan en un fichero $.csv$ y se almacenan en una carpeta para su posterior tratamiento. El estudio se centra en la rama principal del halo, a la cual otros halos menores se le han ido fusionando a lo largo de la historia. \\

Se representa gráficamente cada rango de masa de la \textit{Tabla} ~\ref{tab:tabla2} de la siguiente manera. Se dispone la masa crítica de cada halo normalizada con la masa a \textit{redshift} nulo en frente del inverso del factor de escala, ambos ejes en escala logarítmica decimal. 

\subsection{Resultados: Materia oscura}
\label{subsec:2_B}

En primer lugar, cabe señalar que el crecimiento de los halos se realizan mediante fusiones. Por tanto, el crecimiento ocurre en un instante de tiempo, es decir, en un \textit{snapNum} determinado, y no tiene un crecimiento continuo. Sin embargo, debido a que la representación escalonada de los subhalos dificulta la visualización, en la \textit{Figura} ~\ref{fig:fig3} no se añade. \\

De manera previa a estudiar la evolución de los subhalos para cada rango de masa en la \textit{Figura} ~\ref{fig:fig3}, y con el objetivo de evidenciar la influencia de la masa en la verificación de la evolución según la simulación, se presenta en la \textit{Figura} ~\ref{fig:fig1} una comparación general entre los cinco rangos de masas. Se calcula la media de las masas de los subhalos de cada rango y para cada redshift. Las escalas de los ejes son logarítmicas, luego cada intervalo representa la variación en órdenes de magnitud. \\

\begin{figure}[H]
\centering
\includegraphics[width=0.5\textwidth]{{Averages}.png} 
\caption{\textbf{Evolución de los halos} promedio de los rangos de la \textit{Tabla} ~\ref{tab:tabla2} para el caso de materia oscura. Código de colores: morado-Rango 1, rojo-Rango 2, verde-Rango 3, amarillo-Rango 4 y azul-Rango 5}
\label{fig:fig1}
\end{figure}

Se observa claramente que los halos más masivos tienen una historia más antigua al llegar a redshifts más elevados. Hay un crecimiento inicial considerable y más homogéneos en los tres primeros rangos (morado, rojo y verde) y un aplanamiento de su curva en redshift más cercanos a cero. Además, el rango que representa la curva amarilla no llega a un redshift muy avanzado. Las limitaciones del modelo se observan cláramente en el rango representado por la curva azul, el menos masivo, el cual pone en evidencia la validez de la simulación para rangos de masa tan pequeños. Se hablará de las limitaciones de la simulación del \textbf{Milenio}\cite{6} en la \textit{Sección} ~\ref{sec:4}. \\

La \textit{Figura} ~\ref{fig:fig1} representa la evolución de los halos para cada rango de la \textit{Tabla} ~\ref{tab:tabla2}. En general, se observan intervalos de crecimiento negativo, lo cuál no tiene una explicación física congruente. Estos decrecimientos son una consecuencia de las limitaciones del modelo y se explican entendiendo que, para un \textit{SnapNum} determinado, halos más pequeños salen fuera del radio del virial del halo principal, dejando de ser considerados, sustrayendo así la masa correspondiente. Posteriormente, este decrecimiento se recupera gracias al colapso gravitacional. Los halos que habían salido vuelven a incluirse en el espacio que abarca el radio del virial del halo principal. \\

Si uno se detiene en cada gráfica dentro de la \textit{Figura} ~\ref{fig:fig3}, se pueden hacer los siguientes comentarios. En \textit{(a)}, se aprecia un crecimiento inicial a alto redshift con un número considerable de fusiones entre los distintos subhalos, visualizando un aplanamiento de la curva con una dismunución del número de fusiones a medida que se avanza hacia redshifts más pequeños. En \textit{(b)} y en \textit{(c)}, uno verifica que el crecimiento inicial es menor fijándose en el eje de abscisas, cuyos intervalos se han reducido, por lo que la curva tiene un aplanamiento más temprano. La principal diferencia entre \textit{(b)} y \textit{(c)} es que en el tercer rango de masas, no se llega a un redshift tan elevado. En \textit{(d)}, fijándose de nuevo en el eje de abscisas, el crecimiento es mucho más tardío y menos acentuado. Finalmente, en \textit{(e)}, se pone en evidencia las limitaciones de la simulación fuertemente. La mayoría de los subhalos que se representan, tienen intervalos donde sus valores durante su crecimiento son mucho mayores a sus valores finales, lo cual no tiene consistencia física. \\



\subsection{Resultados: Materia bariónica}
\label{subsec:2_C}

De la misma manera que en la \textit{Sección} ~\ref{subsec:2_B}, el crecimiento es instantáneo pero, para una mejor visualización, la \textit{Figura} ~\ref{fig:fig5} no se representa de manera escalonada. \\

Análogamente, antes de analizar la \textit{Figura}~\ref{fig:fig5} en la que se recogen la evolución de cada conjunto de subhalos para cada rango de masas de la \textit{Tabla}~\ref{tab:tabla2}, se presenta en la \textit{Figura}~\ref{fig:fig2} un promedio de las masas para cada rango y para cada redshift. Igualmente, las esalas son logarítmicas, con lo que la variación entre cada intervalo corresponde a un cambio en el orden de magnitud. \\

El crecimiento de los halos más masivos se prolonga hacia redshift más grandes, por lo tanto su historia de fusiones es más antigua. El aplanamiento de la curva de crecimiento según se avanza hacia redshift más pequeños también es visible en la gráfica. Y, de la misma manera que en la \textit{Figura}~\ref{fig:fig1}, las limitaciones del modelo se hacen visible en la curva azul, correspondiente al promedio del grupo de subhalos del rango más pequeño de masa. Sin embargo, la principal diferencia con el caso de la materia bariónica de la \textit{Sección}~\ref{subsec:2_B}, es que el crecimiento no es igual de homogéneo. La explicación se contará en detalle con la explicación relativa a la \textit{Figura}~\ref{fig:fig5}. \\

\begin{figure}[H]
\centering
\includegraphics[width=0.5\textwidth]{{Averages_baryons}.png} 
\caption{\textbf{Evolución de los halos} promedio de los rangos de la \textit{Tabla} ~\ref{tab:tabla2} para el caso de materia bariónica. Código de colores: morado-Rango 1, rojo-Rango 2, verde-Rango 3, amarillo-Rango 4 y azul-Rango 5}
\label{fig:fig2}
\end{figure}




% \subsubsection{Evolución de los halos del Rango 2 (ver \textit{Tabla} ~\ref{tab:tabla2})}

% \subsubsection{Evolución de los halos del Rango 3 (ver \textit{Tabla} ~\ref{tab:tabla2})}

% \subsubsection{Evolución de los halos del Rango 4 (ver \textit{Tabla} ~\ref{tab:tabla2})}

% \subsubsection{Evolución de los halos del Rango 5 (ver \textit{Tabla} ~\ref{tab:tabla2})}






















