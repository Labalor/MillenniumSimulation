\documentclass[a4paper, amsfonts, amssymb, amsmath, reprint, showkeys, nofootinbib, twoside]{revtex4-1}
\usepackage[english]{babel}
\usepackage[utf8]{inputenc}
\usepackage{verbatim}
\usepackage[colorinlistoftodos, color=green!40, prependcaption]{todonotes}
\usepackage{amsthm}
\usepackage{mathtools}
\usepackage{physics}
\usepackage{xcolor}
\usepackage{graphicx}
\usepackage{subcaption}
\usepackage[left=23mm,right=13mm,top=35mm,columnsep=15pt]{geometry} 
\usepackage{adjustbox}
\usepackage{placeins}
\usepackage[T1]{fontenc}
\usepackage{lipsum}
\usepackage{csquotes}




\usepackage[pdftex, pdftitle={Article}, pdfauthor={Author}]{hyperref} % For hyperlinks in the PDF
%\setlength{\marginparwidth}{2.5cm}
\usepackage{import}
\usepackage{booktabs}

% tiny scriptsize foottesize small rmalsize large Large LARGE huge Huge




\usepackage{vmargin} %Para modificar los márgenes
\setmargins{2.5cm}{1.5cm}{16.5cm}{23.42cm}{10pt}{1cm}{0pt}{2cm}
\definecolor{ceruleanblue}{rgb}{0.16, 0.32, 0.75}
\usepackage{tocloft}
\renewcommand\cftsecafterpnum{\par\addvspace{6pt}}

\usepackage{booktabs} % toprule midrule bottomrule (lineas en las tablas)
\usepackage{array} 
\usepackage{natbib} % quitar título predeterminado de "The Bibliography"
\usepackage{braket} % Dirac notation
\usepackage{amsmath} % Boxed



%\usepackage[pdftex, pdftitle={Article}, pdfauthor={Author}]{hyperref} % For hyperlinks in the PDF
%\setlength{\marginparwidth}{2.5cm}

\usepackage{import}
\usepackage{float} %[h] here,[t] top page,[b] bottom, [H] exactly here (especial del paquete) 
\usepackage{colortbl}
\usepackage{booktabs}
\bibliographystyle{apsrev4-1}


\begin{document}
\renewcommand{\tablename}{\textit{Tabla}}
\renewcommand{\figurename}{\textit{Figura}}

\title{Análisis de árboles de fusiones en simulaciones numéricas}

\author{Víctor Rufo, Yuan Bernete y Luis Abalo}
    \email[Víctor Rufo, Juan Bernete y Luis Abalo: ]{labalo@ucm.es, jbernete@ucm.es y vrufo@ucm.es}
    \affiliation{Universidad Complutense de Madrid, Departamento de Astrofísica, Madrid, España}

\date{\today} % Leave empty to omit a date

\begin{abstract}
En el presente artículo se estudia la evolución de los halos de materia oscura y las galaxias utilizando la \textbf{simulación del Milenio} como base de datos. Dicho análisis se divide en dos partes: una primera parte en la que se realiza el estudio de la evolución y las fusiones en halos de rangos de masas diferentes para su posterior comparación y pequeño estudio estadístico. En la segunda parte, con el objetivo de poder estudiar la evolución de las galaxias albergadas en los subhalos, se considera la materia bariónica y se estudia el color a través de las bandas B y V. El objetivo principal del artículo que se presenta es comprobar si el comportamiento de los halos y galaxias y la descripción de los mecanismos y eventos que conciernen se pueden explicar a través de la simulación.

\end{abstract}

\keywords{evolución: galaxias, halos: materia oscura y bariónica, color: excesos, formación: galaxias y estrellas}

\maketitle


\import{sections/}{seccion01.tex} 
\import{sections/}{seccion02.tex}
\import{sections/}{seccion03.tex}
\import{sections/}{seccion04.tex}

% \import{sections/}{agradecimientos.tex}
\import{sections/}{bibliography.tex}
\newpage
\import{sections/}{apendice01.tex}


\end{document}

