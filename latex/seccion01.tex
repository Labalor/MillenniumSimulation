\section{Introducción} 
\label{sec:1} % no poner acentos

El análisis de la evolución de los halos de materia oscura y galaxias se ha realizado a partir de la base de datos en la aplicación web de \textbf{Virgo - Millennium}\cite{6}, anunciada en (\textit{G. Lemson $\&$ the Virgo Consortium, 2006}\cite{5}), usando \textit{Structured Query Language} (SQL). De esta manera, se accede a todas las propiedades de las galaxias y halos así como a la relación espacial y temporal entre ellos y su entorno. El estudio esta basado en el modelo de \textit{Springel et al., 2005}\cite{1}, el cual hace predicciones sobre el crecimiento jerárquico de las estructuras a través de la inestabilidad gravitatoria. Para el caso de las propiedades de las galaxias, el modelo es el de \textit{G. De Lucia and J. Blaizot, 2007}\cite{2}.\\ 

 Los \textbf{parámetros y condiciones iniciales} de la simulación se basan en el modelo cosmológico $\Lambda$CDM. Son los establecidos en la \textit{Tabla} ~\ref{tab:tabla1}: 
 
\begin{table}[H]
\begin{center}
\begin{tabular}{c|c|c|c|c|c}
\toprule
\cellcolor[gray]{0.9}\large{$\Omega_m$} & \cellcolor[gray]{0.9}\large{$\Omega_b$} & \cellcolor[gray]{0.9}\large{$\Omega_\Lambda$} & \cellcolor[gray]{0.9}\large{$h$} & \cellcolor[gray]{0.9}\large{$n$}& \cellcolor[gray]{0.9}\large{$\sigma_8$}\\
\midrule
0.25 & 0.0045 & 0.75 & 0.73 & 1 & 0.9 \\
\bottomrule
\end{tabular}
\end{center}
\caption{\textbf{Parámetros y condiciones iniciales}: $\Omega_m = \Omega_{dm} + \Omega_b$ es la densidad total de materia, $\Omega_b$ es la densidad de bariones y $\Omega_\Lambda$ es la densidad de materia oscura, las tres en unidades de la densidad crítica para un universo cerrado, $h$ es la constante de Hubble, $n$ es el índice del espectro de perturbaciones iniciales, y $\sigma_8$ es la amplitud de las perturbaciones (\textit{Springel et al., 2005}\cite{1})}
\label{tab:tabla1}
\end{table}

Los datos utilizados pertenecen a la simulación milli-Millenium (millimill), que es una versión reducida en tamaño y número de partículas de la simulación del Milenio. El volumen de la simulación es una caja de 62.5 $h^{-1}Mpc$ de lado, con N = $270^{3}$ partículas de masa $8.6\times 10^8$ $h^{-1}M_\odot$ (\textit{Springel et al., 2005}\cite{1} \cite{7}). La simulación guarda 64 \textit{snapshots}, que se corresponden con el tiempo de la simulación, numeradas del 0 al 63, siendo esta última el presente. \\

Las estructuras de la simulación son identificadas por el algoritmo SUBFIND (\textit{Springel et al., 2001}\cite{8}) junto con el algoritmo \textit{Friends of Friends} (FoF). FoF se basa en la separación entre partículas, conecta aquellas que estén a una distancia menor que 0.2 veces la distancia media entre partículas. Los grupos FoF son divididos en subestructuras identificadas por SUBFIND como regiones sobredensas locales, que si cumplen ciertas características gravitaciones y tienen al menos 20 partículas, son consideradas un subhalo. Al multiplicar este número mínimo de partículas para generar una estructura por la masa de una partícula, se obtiene que la resolución de esta simulación es $1.7\times 10^{10}h^{-1}M_\odot$. Nótese que tras esta clasificación quedan partículas dentro de halos y también fuera de ellos. \\ 

En la \textit{Sección}~\ref{sec:2} se estudian los árboles de fusión en halos de rangos de masas diferentes. Se utilizan los datos de la $m_{crit200}$ (la masa dentro del radio donde FoF tiene una sobredensidad 200 veces mayor que la densidad crítica de la simulación) y el \textit{redshift} ($z=0$ implica un \textit{SnapNum = 63}, última \textit{snapshot} de la simulación). Se obtienen los árboles de fusión de 10 subhalos dentro de los 5 rangos de masa analizados. Posteriormente, se realiza el mismo estudio para galaxias, teniendo en cuenta solo la materia bariónica, y se comparan ambos resultados.\\ 

En la \textit{Sección}~\ref{sec:3} se realiza un estudio de la evolución del color y la masa de tres galaxias escogidas, representando estos parámetros en sus respectivos árboles de fusiones. 

 























