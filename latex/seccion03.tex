\section{Árbol de fusiones} 
\label{sec:3} % no poner acentos

\subsection{Procedimiento}
\label{subsec:3_A}

La simulación de \textbf{Virgo - Millennium}\cite{6} también permite el análisis de un subhalo en concreto para el estudio de la formación y evolución de las distintas galaxias a lo largo de toda su historia hasta redshift cero. En este caso, se utiliza G2 para extraer la información relativa a las galaxias con los IDs utilizados en H2 para cuatro subhalos. \\

Los cuatro halos que se seleccionan pertenecen al Rango 4 de la \textit{Tabla}~\ref{tab:tabla2}, con masas entre $10^{12}h^{-1}M_{\odot}$ y $10^{13}h^{-1}M_{\odot}$. Este rango se escoge porque el número esperado de fusiones es el más adecuado para visualizar correctamente en gráficas. Masas mayores tendrían muchas más fusiones y el analisis se dificultaría, y masas menores se ven mucho más afectadas por los límites de la simulación.\\

Para la representación de la masa, se tiene en cuenta la masa en forma de estrellas, la masa del gas caliente y el gas frío. La masa se representa en la \textit{Figura}~\ref{fig:fig8} mediante el tamaño de los puntos. Para la representación del color B-V, se restan las magnitudes dadas por la simulación en dichas bandas en cada punto y el resultado se da en un eje de color. \\

En la \textit{Figura}~\ref{fig:fig7} se representa el redshift frente a la coordenada espacial X y el color B-V queda reflejado en los colores de cada gráfica. En la \textit{Figura}~\ref{fig:fig8} se representa el redshift frente al identificador de cada elemento del árbol, teniendo en cuenta el color B-V y, además, la evolución de la masa reflejada en el tamaño de los puntos. En la \textit{Figura}~\ref{fig:fig9}, se recoge una representación tridimensional de cada subhalo, recogiendo dos coordenadas espaciales y el redshift, en los ejes, y el color B-V. \\

\subsection{Resultados}
\label{subsec:3_B}

En las \textit{Figuras}~\ref{fig:fig7}, \ref{fig:fig8} y \ref{fig:fig9} del Apéndice se observan las fusiones que se producen a medida que disminuye el \textit{redshift}. La evolución en masa se muestra en la \textit{Figura}~\ref{fig:fig8}. La masa se concentra, fundamentalmente, en la rama principal (rama situada a la derecha), aunque también se reparte una pequeña parte en ramas secundarias. La masa aumenta según se evoluciona hacia el presente y, respecto al color B-V, las galaxias se vuelven más rojas con el tiempo. Únicamente en las galaxias correspondientes a \textit{redshifts} más altos presentan un color más azul. \\

El color de las galaxias tiene una fuerte relación con su formación estelar. Las estrellas más calientes y, por tanto, más azules, tienen tiempos de vida cortos, luego su presencia es un indicador de que se están formando estrellas en la galaxia. Mientras que las estrellas rojas viven mucho más tiempo, por lo que una alta proporción de estas indica una población estelar envejecida y sin formación. Generalmente las galaxias espirales e irregulares, que tienen mucho gas para formar estrellas, son azules, y las galaxias elípticas resultantes de múltiples fusiones son rojas y sin suficiente gas. Esto cuadra con los datos obtenidos de la simulación, donde el paso del tiempo y el crecimiento en masa enrojece a las galaxias. \\ 

Sin embargo, el enrojecimiento no es completamente uniforme. En algunos casos se puede observar que regresa ligeramente al azul tras algunas fusiones. Este efecto es el \textit{"collisional starburst"}, un aumento repentino de la formación estelar tras una fusión por la caída de grandes cantidades de gas. La modelación de este efecto en la simulación se basa en el trabajo de \textit{Sommerville et al. (2001)} \cite{3}. Esta formación de nuevas estrellas viene acompañada de una eyección del gas frío remanente modelada por \textit{Croton et al. (2006)} \cite{10}.


