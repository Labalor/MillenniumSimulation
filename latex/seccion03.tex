\section{Árbol de fusiones} 
\label{sec:3} % no poner acentos

\subsection{Procedimiento}
\label{subsec:3_A}

La simulación de \textbf{Virgo - Millennium}\cite{6} también permite el análisis de un subhalo en concreto para el estudio de la formación y evolución de las distintas galaxias a lo largo de toda su historia hasta redshift cero. En este caso, se utiliza G2 para extraer la información relativa a las galaxias con los IDs utilizados en H2 para cuatro subhalos. \\

Los cuatro halos que se seleccionan pertenecen al Rango 4 de la \textit{Tabla}~\ref{tab:tabla2}, con masas entre $10^{12}h^{-1}M_{\odot}$ y $10^{13}h^{-1}M_{\odot}$.ya en rangos más masivos existe un número más elevado de subhalos sin galaxias. \\

Para la representación de la masa, se tiene en cuenta la masa en forma de estrellas, la masa del gas caliente y frío y la masa del bulbo, en el caso de que tuviera. Notar que no se representa la masa, sino que a través del tamaño de los puntos de la \textit{Figura}~\ref{fig:fig8}, se ilustra la dependencia. Para la representación del color B-V, se restan las magnitudes simuladas en cada punto. \\

En la \textit{Figura}~\ref{fig:fig7} se representa el redshift frente a la coordenada espacial X y el color B-V queda reflejado en los colores de cada gráfica. En la \textit{Figura}~\ref{fig:fig8} se representa el redshift frente al identificador de cada elemento del árbol, teniendo en cuenta el color B-V y, además, la evolución de la masa reflejada en el tamaño de los puntos. En la \textit{Figura}~\ref{fig:fig9}, se recoge una representación tridimensional de cada subhalo, recogiendo dos coordenadas espaciales y el redshift, en los ejes, y el color B-V. \\

\subsection{Resultados}
\label{subsec:3_B}

En la \textit{Figura}~\ref{fig:fig8} se observan las fusiones que se producen a medida que se produce la evolución temporal, es decir, creciendo en el eje de abscisas. La masa se concentra, fundamentalmente, en la rama principal (rama situada a la derecha), salvo en la figura (a), que también se reparte en la rama secundaria. La masa aumenta según se evoluciona hacia redshift cero y, respecto al color B-V, las galaxias se vuelven más rojas con el tiempo. Únicamente en las galaxias correspondientes a redshift más altos presentan una mayor luminosidad en el UV. \\



