\section{Conclusiones} 
\label{sec:4}

La \textbf{simulación del Milenio}\cite{6}, anunciada en (\textit{G. Lemson $\&$ the Virgo Consortium, 2006}\cite{5}) funciona bien para halos con una masa elevada y, sin embargo, funciona con mucha inexactitud para rangos de masa cerca al valor límite de la simulación. De esta manera, se decide despreciar los halos menos masivos, debido a la cantidad de errores e imprecisiones que acumulan. \\

Es interesante fijarse que el aplanamiento de la curva de crecimiento para los halos menos masivos se produce de manera cronológica antes que para los halos más masivos. Además, la historia temporal presenta un menor crecimiento inicial, con un menor número de fusiones, respecto a los halos más masivos. \\

Respecto a las limitaciones de la simulación, en primer lugar, se señala el problema de reproducir un comportamiento correcto con halos de masas cercanas al límite de la simulación, cerca a $10^{10}M_{\odot}$. En segundo lugar, la cantidad de partículas es excasa en halos pequeños para representar las estructuras de manera correcta. \\ 


