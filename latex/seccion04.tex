\section{Conclusiones} 
\label{sec:4}

La \textbf{simulación del Milenio}\cite{6}, por \textit{Springel et al. (2005)} \cite{1}, anunciada en (\textit{G. Lemson $\&$ the Virgo Consortium, 2006}\cite{5}) aporta buenas predicciones para halos con una masa elevada y, sin embargo, funciona con mucha inexactitud para rangos de masa cerca al valor límite de la simulación. De esta manera, no es conveniente estudiar mediante esta simulación los halos con masas entre dicho límite y el siguiente orden de magnitud, debido a la cantidad de errores e imprecisiones que acumulan. \\

El límite de la simulación viene dado por la masa de cada partícula de la simulación y por el algoritmo que utiliza para el reconocimiento de estructuras, SUBFIND (\textit{Springel et al., 2001} \cite{8}). Con una masa por partícula de $8.6\times10^{8}h^{-1}M_\odot$ y un mínimo de 20 partículas para generar una estructura, el límite se establece en $1.7\times10^{10}h^{-1}M_\odot$. Este límite genera problemas para reproducir un comportamiento correcto con halos de masas entre  $1.7\times10^{10}h^{-1}M_{\odot}$ y $1.7\times10^{11}h^{-1}M_\odot$, ya que la cantidad de partículas en estos halos no es suficiente para una buena descripción de sus procesos físicos, ni permite conocer su historia a altos \textit{redshifts}, donde su masa estaría por debajo del límite y, por tanto, quedaría fuera del alcance de la simulación. \\

En los casos alejados al límite de la simulación es interesante fijarse que el aplanamiento de la curva de crecimiento para los halos menos masivos se produce cronológicamente antes que para los halos más masivos. Además, la historia temporal presenta un menor crecimiento inicial, con un menor número de fusiones, respecto a los halos más masivos. La tendencia general es la de un crecimiento jerárquico por fusiones y absorción del medio intergaláctico (partículas que quedan fuera de halos), tal como se espera. \\

Al añadir el trabajo de \textit{G. De Lucia and J. Blaizot (2006)} \cite{2}, se pueden estudiar también las propiedades de las galaxias centradas en los halos. Su comportamiento en masa sigue también una crecimiento jerárquico, pero con mayores alteraciones producidas por diferentes efectos físicos modelados, como su desplazamiento del centro del halo en las fusiones, el \textit{"collisional starburst" (Somerville et al., 2001)} \cite{3} o la eyección de gas por el \textit{feedback} de la formación estelar \textit{(Croton et al., 2006)} \cite{10}. \\

Finalmente, se comprueban las buenas predicciones de \textit{G. De Lucia and J. Blaizot (2006)} \cite{2} en el envejecimiento de las galaxias al estudiar el color B-V en los árboles de fusiones. Las galaxias jóvenes y con alta formación estelar son, tal como se espera, más azules, y las envejecidas sin formación estelar, más rojas. El efecto del \textit{"collisional starburst"} también es visible en un ligero regreso al azul que se produce tras algunas fusiones de galaxias.
